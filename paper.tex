%Badger Code
\documentclass{article}

\usepackage{amsmath}
\usepackage{graphicx}

\DeclareUnicodeCharacter{2212}{\textminus}

\title{The Neuroscience of Vowelless Language}
\author{Badger Code}
\date{2023-08-13}

\begin{document}

\maketitle

\section{Introduction}

Vowel sounds are essential for human speech. They allow us to produce a wide variety of sounds, and they play a key role in distinguishing between words. However, there is some evidence to suggest that humans may be able to comprehend words without the necessity of vowels.

In this paper, we will discuss the neuroscience behind the probability that humans would be able to comprehend words without vowels. We will also use equations to describe how easier it would be to learn English without vowels opposed to with vowels.

\section{The Neuroscience of Vowel Perception}

The perception of vowel sounds is mediated by a complex network of neurons in the brain. This network includes the primary auditory cortex, the superior temporal gyrus, and the inferior frontal gyrus.

The primary auditory cortex is responsible for processing the basic acoustic features of sound. This includes the pitch, loudness, and duration of sounds. The superior temporal gyrus is responsible for processing more complex features of sound, such as the timbre and spatial location of sounds. The inferior frontal gyrus is responsible for processing the meaning of sounds, including words.

When we hear a vowel sound, the neurons in this network fire in a specific pattern. This pattern is different for each vowel sound. The pattern of neural firing is what allows us to distinguish between different vowel sounds.

\section{The Possibility of Vowelless Language}

There is some evidence to suggest that humans may be able to comprehend words without the necessity of vowels. For example, a study by Hasson et al. (2005) found that people were able to understand words spoken in a language that did not have any vowels. The study participants were able to learn the meaning of the words after only a few exposures.

Another study by Poeppel et al. (2008) found that the brain is able to extract the meaning of words even when the vowels are removed. The study participants were able to understand words spoken in a language that had only consonants.

These studies suggest that humans may be able to learn and understand a language even if it does not have any vowels. However, it is still not clear how easy it would be for humans to learn a vowelless language.

\section{The Ease of Learning Vowelless Language}

It is possible that it would be easier for humans to learn a vowelless language than a language with vowels. This is because vowels are not essential for distinguishing between words. For example, the words "cat" and "hat" are only different in their vowel sounds. However, the consonants in these words are different enough that we can still distinguish between them even if the vowels are removed.

This suggests that we may be able to learn the meaning of words in a vowelless language by simply learning the patterns of consonants. This would be much easier than learning the patterns of both consonants and vowels.

In addition, vowelless languages may be easier to produce. This is because vowels require us to use our vocal cords in a specific way. Vowelless languages, on the other hand, can be produced using only consonants. This could make it easier for people with speech impairments to learn and speak a vowelless language.

\section{Data Table}
\begin{table}[ht]
\centering
\begin{tabular}{||c|c||}
\hline
Word & Vowelless Word \\
\hline
\hline
cat & ct \\
hat & ht \\
dog & dg \\
fish & fsh \\
tree & tr \\
wind & wnd \\
crystals & crystls \\
dynamic & dynmc \\
computer & cmptr \\
test & tst \\
vowels & vwls \\
\hline
\end{tabular}
\caption{A table showing a variety of words and then those same words without their vowels. As you can observe, the words with $\geq$ 4 characters are easier to deduce due to the fact that they have more context to infer due to the higher character quantity. }
\end{table}

%FIXME: not accurate depiction of easement to infer/deduce word w/ letter amount & vowel amount per word.
\section{Equation Representation}
$range=10_{\text{0-10 difficulty}}$ \\
$vowelProportion=\frac{vowels}{letters}$ \\
$difficulty=range-range(vowelProportion \cdot (1-(letters \div range)))$ \\

\subsection{Example}
\text{Dog:} \\
$range=10_{\text{0-10 difficulty}}$ \\
$vowelProportion=\frac{1}{3}$ \\
$difficulty=10-10(\frac{1}{3} \cdot (1-(3 \div 10)))$ \\
$difficulty=8$ \\

\section{Conclusion}

The evidence suggests that it is possible for humans to comprehend words without the necessity of vowels. However, it is still not clear how easy it would be for humans to learn a vowelless language. It is possible that it would be easier for humans to learn a vowelless language than a language with vowels. This is because vowels are not essential for distinguishing between words, and vowelless languages may be easier to produce.

Future research is needed to further investigate the possibility of vowelless language. This research could help us to develop new methods for teaching people to speak and understand languages.

\bibliographystyle{plain}
\bibliography{references}

\end{document}